\documentclass[10pt,a4paper]{article}
\usepackage[italian]{babel}
\usepackage{geometry}
\usepackage{graphicx}
\usepackage{mathtools}
\usepackage{hyperref}
\usepackage{amsmath}
\usepackage{cleveref}
\usepackage{rotating}
\usepackage{pdflscape}
\usepackage{xspace}
\usepackage{subcaption}
\usepackage{array}
\usepackage{soul}
\usepackage{ragged2e}
\usepackage[shortlabels]{enumitem}
\usepackage{multirow}
\usepackage{fancyvrb}
\usepackage[dvipsnames]{xcolor}
\crefname{section}{sezione}{sezioni}
\renewcommand{\underline}{\ul}
\newcolumntype{x}[1]{>{\RaggedLeft\hspace{0pt}}p{#1}}
\newcolumntype{y}[1]{>{\RaggedRight\hspace{0pt}}p{#1}}
\newcolumntype{z}[1]{>{\Centering\hspace{0pt}}p{#1}}
\newtheorem{theorem}{Teorema}
\newcommand{\bitem}[1]{\item\textbf{#1:}\xspace}
\newcommand{\er}[2]{\text{E}({#1,[#2]})}
\newcommand{\dr}[2]{\text{D}({#1,#2})}
\newcommand{\di}[2]{\textit{D}_{K#1}(#2)}
\newcommand{\p}[2]{\text{\uppercase{#1}}_\textit{{#2}}}
\newcommand{\pu}[1]{\p{PU}{#1}}
\newcommand{\pr}[1]{\p{PR}{#1}}
\newcommand{\ks}{\p{K}{s}}
\newcommand{\ki}[1]{{\p{K}{#1}}}
\newcommand{\ts}[1]{\textit{TS}_{#1}}
\newcommand{\lft}[1]{\textit{Lifetime}_{{#1}}}
\newcommand{\auth}[1]{\textit{Authenticator}_{{#1}}}
\newcommand{\tk}[1]{\textit{Ticket}_{{#1}}}
\newcommand{\uc}[1]{\text{\uppercase{#1}}}
\newcommand{\ei}[2]{\textit{E}_{K#1}(#2)}
\newcommand{\n}[1]{\p{N}{#1}}
\newcommand{\id}[1]{\p{ID}{#1}}
\newcommand{\crt}[1]{\ll#1\gg}
\newcommand{\conc}{\parallel}
\newcommand{\tab}{\colon\quad}
\newcommand{\hs}[1]{\text{H}(#1)}
\newcommand{\ad}[1]{\p{AD}{#1}}
\newcommand{\dsa}[2]{\text{DSA}_{#1}(#2)}
\newcommand{\sa}[1]{\xrightarrow{\quad#1\quad}}
\newtheorem{definition}{Definizione}
\newtheorem{assumption}{Assunzione di sicurezza}
\title{Lecture: protocollo di Needham-Schroeder}
\author{Riccardo Torre}
\begin{document}
	\maketitle
	\begin{align*}
		A\to S \acolon A,B\\
		S\to A\acolon \ki{AB}\\
		B\to A\acolon \ki{AB},A
	\end{align*}
	\begin{assumption}
		L'avversario è capace di ascoltare tutti i messaggi in un protocollo di sicurezza. 
		\end{assumption}
		Dunque è necessaria la crittografia dei messaggi.
		\separate
		\begin{assumption}[Perfect cryptography assumption]
			L'attaccante in generale non può rompere la crittografia.\label{asp:perfect-critto}
		\end{assumption}
		\begin{align*}
			A\to S\acolon A,B\\ S\to A\acolon \ei{AS}{\ki{AB}},\ei{BS}{\ki{AB}}\\
			A\to B\acolon \ei{BS}{\ki{AB}},A \qquad\longrightarrow\qquad \text{C intercetta il messaggio}\\
			C \to B\acolon \ei{BS}{\ki{AB}},D
		\end{align*}
		C'è un attacco peggiore di questo. In questo caso, C è autenticato (è un insider del sistema, perché S accetta la sua richiesta).
		\begin{align*}
			A\to S\acolon A,B\qquad\text{C intercetta il messaggio}\\
			C\to S \acolon A,C\\
			S\to C\acolon \ei{AS}{\ki{AC}},\ei{CS}{\ki{AC}}\\
			C\to A\acolon \ei{AS}{\ki{AC}},\ei{CS}{\ki{AC}}\qquad\text{A non si accorge che C sta nel mezzo}\\
			A\to C\acolon \ei{CS}{\ki{AC}},A
			\end{align*}
			Le chiavi non sono legate crittograficamente all'identità di chi deve farne uso (\textbf{binding attack}). L'attacco appena descritto è inoltre un \textbf{man-in-the-middle} e uno \textbf{spoofing attack}.
			\begin{assumption}
				L'avversario può essere un partecipante legittimo del protocollo (un \textit{insider}), o una parte esterna (un \textit{outsider}) o una combinazione delle precedenti.
			\end{assumption}
			\separate
			\begin{align*}
				A\to S\acolon A,B\\
				S\to A\acolon \ei{AS}{\ki{AB},B},\ei{BS}{\ki{AB},A}\\
				A\to B\acolon \ei{BS}{\ki{AB},A}
			\end{align*}
			Questo fix lega crittograficamente la chiave all'entità che deve utilizzarla. Ma questo protocollo è ancora vulnerabile ad un \textbf{replay attack}.
			\begin{align*}
				A\to S \acolon A,B\qquad\longrightarrow\text{C intercetta la richiesta, la blocca e si finge di essere il server}\\ C\to A \acolon \ei{AS}{\ki{AB}', B},\ei{BS}{\ki{AB}',A}\qquad\longrightarrow\substack{\text{dove $\ki{AB}'$ è una vecchia chiave di sessione}\\ \text{intercettata da $C$ in una vecchia interazione }\\\text{tra $A$ e $S$}}\\ A\to B\acolon \ei{BS}{\ki{AB}',A}
			\end{align*}
			La risposta in seguito alla richiesta di A ad S è di una vecchia interazione tra A e S, solo che A non se ne accorge perché manca questa informazione nel protocollo. Bisogna legare le richieste alle risposte con i \textbf{nonce} o \textbf{challenge-responce}. Arriviamo così al protocollo di \ns.
			\separate
			\paragraph{Il protocollo di \ns}
			\begin{align*}
				A\to S\acolon A,B,\n{A}\\ S\to A\acolon \ei{AS}{\ki{AB}, B, \n{A},\ei{BS}{\ki{AB},A}}\\A \to B \acolon \ei{BS}{\ki{AB},A}\\B\to A\acolon \ei{AB}{\n{B}}\qquad\longrightarrow \text{viene fatto una sorta di acknowledgement}\\ A\to B\acolon \ei{AB}{\n{B}-1}
			\end{align*}
			Anche questo protocollo è vulnerabile al \textbf{replay attack} sul messaggio $B\to A \colon \ei{AB}{\n{B}}$. $A$ ha veramente la garanzia che la chiave è fresh, ma $B$ no! 
			\paragraph{Attacco al protocollo \ns}
			In questo scenario, viene indicata con $\ki{$\overline{AB}$}$ la chiave compromessa, che $C$ utilizza per compiere il \textbf{replay attack} verso $B$.
		\begin{align*}
			A\to S\acolon A,B,\n{A}\\ S\to A\acolon \ei{AS}{\ki{AB}, B, \n{A},\ei{BS}{\ki{AB},A}}\\C \to B \acolon \ei{BS}{\ki{$\overline{AB}$},A}\\B\to C\acolon \ei{$\overline{AB}$}{\n{B}}\qquad\longrightarrow \text{viene fatto una sorta di acknowledgement}\\ C\to B\acolon \ei{$\overline{AB}$}{\n{B}-1}
		\end{align*}
		
		$B$ è convinto che sia stato $A$ a mandargli la $\ki{$\overline{AB}$}$.  \ul{Qui il nonce $\n{B}$ viene inviato dopo che è stata ricevuta la chiave!} Questo non permette di concludere che la chiave ricevuta provenga da un'interazione recente!
		
		Il problema è che $B$ non si accorge che la chiave è vecchia perché il nonce $\n{B}$ non è legato crittograficamente alla chiave $\ki{AB}$ e viene usato più come un acknowledgment che come un qualcosa per verificare la freschezza della chiave di sessione. Invece $A$ può rendersi conto che la chiave di sessione è fresca perché il nonce che ha inviato, $\n{A}$, è legato crittograficamente alla chiave $\ki{AB}$ e non può essere modificato da un attaccante nell'ipotesi che valga l'\cref{asp:perfect-critto}.
		\paragraph{Fix a questo attacco}
		\begin{align*}
			B\to A\acolon B,\n{B}\\
			A\to S \acolon A,B,\n{A},\n{B}\\
			S\to A\acolon \ei{AS}{\ki{AB},B,\n{A}}, \ei{BS}{\ki{AB},A,\n{B}}\\
			A\to B\acolon \ei{BS}{\ki{AB},A,\n{B}}
		\end{align*}
		In questo caso i nonce inviati da $A$ e $B$ vengono legati crittograficamente alla chiave e un attacco come il precedente non può avvenire.
		\separate
		\paragraph{Il protocollo di \ns a chiave pubblica} Questa è la versione di \ns usando chiavi asimmetriche! L'obbiettivo di \ns in generale è di creare una sorta di \textit{ping autenticato}. In questo caso, l'interazione con il serve viene omessa.
		\begin{align*}
			A\to B \acolon \ei{B}{\n{A},A}\\
			B\to A\acolon \ei{A}{\n{A},\n{B}}\\
			A\to B\acolon \ei{B}{\n{B}}
		\end{align*}
		\paragraph{Attacco man-in-the-middle a \ns a chiave pubblica}$C$ è un insider e $A$ non sa che è cattivo. $A$ decide di aprire una connessione con lui. 
		\begin{align*}
			A\to C \acolon \ei{C}{\n{A},A}\\
			C\to B\acolon\ei{B}{\n{A},A}\\
			B\to C\acolon \ei{A}{\n{A},\n{B}}\qquad\longrightarrow\qquad\substack{\text{$C$ non può aprire il messaggio per leggere $\n{B}$}\\\text{perché codificato con $\ki{A}$}}\\
			C\to A\acolon \ei{A}{\n{A},\n{B}}\\
			A\to C\acolon \ei{C}{\n{B}}\\
			C\to B\acolon \ei{B}{\n{B}}
		\end{align*}
		L'attaccante ha fatto uno spoofing attack nei confronti di Bob, spacciandosi per Alice. Inoltre questo è un \textbf{binding attack}.
		\paragraph{Soluzione di Lowe}
		\begin{align*}
			A\to C \acolon \ei{C}{\n{A},A}\\
			C\to B\acolon\ei{B}{\n{A},A}\\
			B\to C\acolon \ei{A}{\n{A},\n{B},B}\qquad\longrightarrow\qquad\substack{\text{$B$ risponde con un messaggio corretto ovvero lega}\\\text{crittograficamente  il nonce da lui prodotto alla sua identità}}\\
			C\to A\acolon \ei{A}{\n{A},\n{B}, B}\qquad\longrightarrow\qquad\text{Bob? Ma io volevo parlare con Charlie!}\\
			&\texttt{\color{red}! Abort del protocollo !}
		\end{align*}
		Il protocollo non è sound.
		\paragraph{Type flaw attack} Questo è un attacco man-in-the-middle condotto da $M$ che consiste in un \textbf{oracle attack} per fare un \textbf{type-flaw attack}.
		\begin{align*}
			M\to B\acolon \ei{B}{\n{A},A}\\
			B\to S\acolon B,A\\
			S\to B \acolon \ei{$S^-1$}{A,\ki{A}}\\
			B\to M\acolon \ei{A}{\n{A},\underbrace{\boxed{\n{B},B}}_{\text{id}}}\\
			M\to A\acolon \ei{A}{\n{A},\boxed{\n{B}, B}}\color{blue}\qquad\longrightarrow\qquad \text{Oracle attack + Type flaw attack!}\\
			A\to M\acolon A,\boxed{\n{B},B}\color{red}\qquad\longrightarrow\qquad \text{$\n{B},B$ viene mandato in chiaro!}\\
			M\to B\acolon \ei{B}{\n{B}}
		\end{align*}
		L'\textbf{oracle attack} consiste nell'inoltrare il messaggio ad $A$ per farselo decifrare ($A$ è l'unico a poterlo fare!). Il \textbf{type flaw attack} invece, consiste nello sfruttare una vulnerabilità nel protocollo, ovvero che i \ul{messaggi non sono tipati}, dunque il nonce di $B$ e l'identificatore di $B$, che sono una stringa di bit, vengono confusi da $A$ per un identificativo perché $A$ pensa che $\n{B},B$ stia aprendo il protocollo con lui, dicendo \textit{``questo è il mio nonce, voglio parlare con $B$''}. Quindi $A$ invia la richiesta di certificato in chiaro. Ma in realtà $A$ sta inviando informazioni che non dovrebbero essere divulgate ossia il nonce di $B$ e l'identificativo di $B$.
		
\end{document}