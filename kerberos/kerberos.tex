\documentclass[10pt,a4paper]{article}
\usepackage[italian]{babel}
\usepackage{geometry}
\usepackage{graphicx}
\usepackage{mathtools}
\usepackage{hyperref}
\usepackage{amsmath}
\usepackage{cleveref}
\usepackage{rotating}
\usepackage{pdflscape}
\usepackage{xspace}
\usepackage{subcaption}
\usepackage{array}
\usepackage{soul}
\usepackage{ragged2e}
\usepackage[shortlabels]{enumitem}
\usepackage{multirow}
\usepackage{fancyvrb}
\usepackage[dvipsnames]{xcolor}
\crefname{section}{sezione}{sezioni}
\renewcommand{\underline}{\ul}
\newcolumntype{x}[1]{>{\RaggedLeft\hspace{0pt}}p{#1}}
\newcolumntype{y}[1]{>{\RaggedRight\hspace{0pt}}p{#1}}
\newcolumntype{z}[1]{>{\Centering\hspace{0pt}}p{#1}}
\newtheorem{theorem}{Teorema}
\newcommand{\bitem}[1]{\item\textbf{#1:}\xspace}
\newcommand{\er}[2]{\text{E}({#1,[#2]})}
\newcommand{\dr}[2]{\text{D}({#1,#2})}
\newcommand{\di}[2]{\textit{D}_{K#1}(#2)}
\newcommand{\p}[2]{\text{\uppercase{#1}}_\textit{{#2}}}
\newcommand{\pu}[1]{\p{PU}{#1}}
\newcommand{\pr}[1]{\p{PR}{#1}}
\newcommand{\ks}{\p{K}{s}}
\newcommand{\ki}[1]{{\p{K}{#1}}}
\newcommand{\ts}[1]{\textit{TS}_{#1}}
\newcommand{\lft}[1]{\textit{Lifetime}_{{#1}}}
\newcommand{\auth}[1]{\textit{Authenticator}_{{#1}}}
\newcommand{\tk}[1]{\textit{Ticket}_{{#1}}}
\newcommand{\uc}[1]{\text{\uppercase{#1}}}
\newcommand{\ei}[2]{\textit{E}_{K#1}(#2)}
\newcommand{\n}[1]{\p{N}{#1}}
\newcommand{\id}[1]{\p{ID}{#1}}
\newcommand{\crt}[1]{\ll#1\gg}
\newcommand{\conc}{\parallel}
\newcommand{\tab}{\colon\quad}
\newcommand{\hs}[1]{\text{H}(#1)}
\newcommand{\ad}[1]{\p{AD}{#1}}
\newcommand{\dsa}[2]{\text{DSA}_{#1}(#2)}
\newcommand{\sa}[1]{\xrightarrow{\quad#1\quad}}
\newtheorem{definition}{Definizione}
\newtheorem{assumption}{Assunzione di sicurezza}
\title{Lecture Kerberos}
\author{Riccardo Torre}
\begin{document}
	\maketitle
	\paragraph{Cos'è}Servizio di autenticazione che fornisce un server di autenticazione centralizzato la cui funzione è di autenticare gli utenti ai server e i server agli utenti. Anche i server sono autenticati tra di loro.
	
	\paragraph{Nota}Kerberos fa uso solo ed \ul{esclusivamente di crittografia simmetrica!!! (DES a 56 bit)}
	
	\paragraph{Requisiti Kerberos}Sicuro, trasparente, scalabile, affidabile.
	\paragraph{Attori}
	\begin{enumerate}
		\bitem{AS = Authentication Server} conosce gli hash delle password di tutti gli utenti e li memorizza in un database centralizzato e crea ogni volta una chiave simmetrica con il client sfruttando l'hash della password come seme da passare a funzioni pseudo random; condivide un'unica chiave con tutti i server;
		\bitem{TGS = Ticket Granting Server}; emette i ticket agli utenti autenticati dall'AS necessari per ottenere un particolare servizio. Il ticket ha una durata (\textit{Lifetime}) ben precisa e può essere riutilizzato. Condivide le chiavi con tutte le risorse (gli application server).
		\bitem{Application server}forniscono un determinato servizio.
	\end{enumerate}
	\paragraph{Ticket}
	\begin{itemize}
		\item Ticket per entrare nel gioco che hanno una validità $Time_1$ ore $\to$ crittografia più forte;
		\item Ticket per utilizzare un servizio che hanno una validità $Time_2$ minuti $\to$ crittografia leggera;
	\end{itemize}
	$$Time_1>Time_2$$
	e 
	$$\lft{1}>\lft{2}$$
	ovvero la durata del ticket per entrare nel gioco è maggiore di quella del ticket per utilizzare un servizio.
	Man mano si usano ticket che durano sempre di meno.
	\paragraph{Authenticator}Serve per verificare che chi sta operando e chi si dichiara sono la stessa persona. Può essere usato una sola volta.
	\paragraph{Legenda}
	\begin{itemize}
		\bitem{C} Client
		\bitem{AS} Authentication Server
		\bitem{TGS}Ticket Granting Server
		\bitem{V}server di un particolare servizio (ad esempio, il server di stampa).
	\end{itemize}
	\paragraph{Note sui messaggi}I messaggi sono 6 in totale. Di seguito vengono riportati alcune note relative ai messaggi (1 a 6)
	\begin{enumerate}[(1)]
		\item C fa autenticazione con il server AS; non trasmette la password perché non deve viaggiare in chiaro (per prevenire da un attacco man-in-the-middle); deve esserci un meccanismo per aggiornare la password
		\item AS restituisce un Ticket; la chiave di sessione non viene trasmessa ma generata da C e da AS utilizzando una funzione basata su DES a cui viene passato H(password)\footnote{hash della password.} come seme.
		\item il ticket è firmato da AS; TGS riceve il ticket, verifica se C ha i permessi per usufruire del servizio fornito da V.
	\end{enumerate}
	\paragraph{Il protocollo}
	\begin{enumerate}[(1)]
		\item $\uc{c}\to\uc{as}\tab \id{c}\conc\id{tgs}\conc \ts{1}$
		\item $\uc{as}\to\uc{c}\tab
		 \er{\ki{c}}{\ki{c,tgs}\conc\id{tgs}\conc\ts{2}\conc\lft{2}\conc\tk{tgs}}$  $$\tk{tgs}=\er{\ki{tgs}}{\ki{c,tgs}\conc\id{c}\conc\ad{c}\conc\id{tgs}\conc\ts{2}\conc\lft{2}}$$
		 \item $\uc{c}\to\uc{tgs}\tab \id{v}\conc\tk{tgs}\conc\auth{c}$
		 \item $\uc{tgs}\to\uc{c}\tab \er{\ki{c,tgs}}{\ki{c,v}\conc\id{v}\conc\ts{4}\conc\tk{v}}$
		 \begin{gather*}
		 	\tk{tgs}=\er{\ki{tgs}}{\ki{c,tgs}\conc\id{c}\conc\ad{c}\conc\id{tgs}\conc\ts{2}\conc\lft{2}}\\
		 	\tk{v}=\er{\ki{v}}{\ki{c,v}\conc\id{c}\conc\ad{c}\conc\id{v}\conc\ts{4}\conc\lft{4}}\\
		 	\auth{c}=\er{\ki{c,tgs}}{\id{c}\conc\ad{c}\conc\ts{3}}
		 \end{gather*}
		 \item $\uc{c}\to\uc{v}\tab \tk{v}\conc\auth{c}$
		 \item $\uc{v}\to\uc{c}\tab \er{\ki{c,v}}{\ts{5}+1}$ (per fare la mutua autenticazione)
		 \begin{gather*}
		 	\tk{v}=\er{\ki{v}}{\ki{c,v}\conc\id{c}\conc \ad{c}\conc \id{v}\conc\ts{4}\conc\lft{4}}\\
		 	\auth{c}=\er{\ki{c,v}}{\id{c}\conc\ad{c}\conc \ts{5}}
		 \end{gather*}
	\end{enumerate}
	\paragraph{Note sul protocollo}Nel messaggio (2) la chiave $\ki{c}$ viene generata usando gli hash della password locali come seme da passare una funzione pseudorandom separatamente sia da $\ki{c}$ che da AS.  La chiave $\ki{c}$ viene cambiata ogni volta che si ripete il protocollo da capo. Le identità vengono legate crittograficamente alle chiavi per evitare \textbf{binding attacks}. Il ticket è qualcosa di codificato con la chiave condivisa fra AS e TGS. In Kerberos v4 la crittografia è annidata. La versione di Needham-Schroeder simmetrica è molto simile. Nel ticket viene inserito anche l'indirizzo di rete per identificare il client nella rete. Nel messaggio (3) viene presentato il ticket assieme ad $\id{v}$ perché non viene specificato all'interno della codifica di $\tk{v}$ (dunque serve ad identificare il server che offre il servizio di cui C vuole usufruire). L'$\auth{c}$ serve al TGS per verificare che di $\tk{tgs}$:\begin{itemize}
		\item $\ts{2}<\ts{3}$;
		\item $\ts{3}<\lft{2}$; 
	\end{itemize} dunque è necessario per garantire la sicurezza del protocollo che i clock siano sincronizzati.
	In generale deve valere che $\ts{i}<\ts{i+1}$ e $\ts{i+1}<\lft{i}$.
	\paragraph{Realms}Kerberos è \textbf{scalabile}, è possibile mettere insieme dei realm, sistemi kerberizzati (AS,TGS e i vari servizi). Si fa uno sharing del database centralizzato di tutte le risorse disponibili localizzato sul Kerberos master computer system. C'è una confederazione di realms. I vari realm hanno una copia del database centralizzato. C nel realm A si autentica con l'AS e ottiene un ticket per presentarlo al TGS di A specificando di voler utilizzare il servizio S del realm B. Dunque il TGS crea un apposito ticket e lo restituisce a C. Questo ticket viene presentato al TGS di B perché è lui che amministra e gestisce i servizi nel realm B (e non il TGS nel realm A). C'è una relazione di fiducia tra i due TGS (TGS realm A con TGS realm B). Quindi il TGS di B restituisce il ticket da presentare al server che offre il servizio S.
	\paragraph{Numero di chiavi nell'autenticazione cross-realms}Supponendo che ci siano $n$ realm, il realm A per poter parlare con gli altri $n-1$ realm ha bisogno di $\frac{n(n-1)}{2}$ chiavi. Questa formula è derivata dal calcolo degli archi di un grafo completo. Se si modellano i realms come i nodi di un grafo completo, e gli archi che collegano i TGS dei vari realms, per ciascun nodo si contano il numero di archi uscenti e i nodi sono $n$. Dato che ogni arco verrà contato due volte (si pensi al caso di un grafo completo composto da due nodi $A\to B$ e $B\to A$) è necessario effettuare la divisione per 2.
	\paragraph{v4 vs v5} AES al posto di DES, ticket lifetime con tuning al posto di ticket da 24 h, qualsiasi net address e non solo IP address, ordinamento dei byte dei messaggi, authentication forwarding che permette di evitare di chiedere un ticket per risorse correlate (ad esempio chiedere il ticket per la stampante e chiedere il ticket per il file da stampare diventa chiedere il ticket per stampare il file), autenticazione interrrealm meno chiavi (meno di $O(n^2)$), non c'è più un encoding annidato del ticket, cioè il ticket è stato separato dal messaggio che lo conteneva.

\end{document}