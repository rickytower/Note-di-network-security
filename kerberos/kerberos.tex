\documentclass[10pt,a4paper]{article}
\usepackage[italian]{babel}
\usepackage{geometry}
\usepackage{graphicx}
\usepackage{mathtools}
\usepackage{hyperref}
\usepackage{amsmath}
\usepackage{cleveref}
\usepackage{rotating}
\usepackage{pdflscape}
\usepackage{xspace}
\usepackage{subcaption}
\usepackage{array}
\usepackage{soul}
\usepackage{ragged2e}
\usepackage[shortlabels]{enumitem}
\usepackage{multirow}
\usepackage{fancyvrb}
\usepackage[dvipsnames]{xcolor}
\crefname{section}{sezione}{sezioni}
\renewcommand{\underline}{\ul}
\newcolumntype{x}[1]{>{\RaggedLeft\hspace{0pt}}p{#1}}
\newcolumntype{y}[1]{>{\RaggedRight\hspace{0pt}}p{#1}}
\newcolumntype{z}[1]{>{\Centering\hspace{0pt}}p{#1}}
\newtheorem{theorem}{Teorema}
\newcommand{\bitem}[1]{\item\textbf{#1:}\xspace}
\newcommand{\er}[2]{\text{E}(\uppercase{#1,#2})}
\newcommand{\dr}[2]{\text{D}(\uppercase{#1,#2})}
\newcommand{\p}[2]{\text{\uppercase{#1}}_\textit{\uppercase{#2}}}
\newcommand{\pu}[1]{\p{PU}{#1}}
\newcommand{\pr}[1]{\p{PR}{#1}}
\newcommand{\ks}{\p{K}{S}}
\newcommand{\ki}[1]{{\p{K}{#1}}}
\newcommand{\ts}[1]{\textit{TS}_{#1}}
\newcommand{\lft}[1]{\textit{Lifetime}_{\uc{#1}}}
\newcommand{\auth}[1]{\textit{Authenticator}_{\uc{#1}}}
\newcommand{\tk}[1]{\textit{Ticket}_{\uc{#1}}}
\newcommand{\uc}[1]{\text{\uppercase{#1}}}
\newcommand{\ei}[2]{\textit{E}_{\ki{#1}}(#2)}
\newcommand{\n}[1]{\p{N}{#1}}
\newcommand{\id}[1]{\p{ID}{#1}}
\newcommand{\crt}[1]{\ll#1\gg}
\newcommand{\conc}{\,||\,}
\newcommand{\hs}[1]{\text{H}(#1)}
\newtheorem{definition}{Definizione}
\newtheorem{assumption}{Assunzione di sicurezza}
\title{Lecture Kerberos}
\author{Riccardo Torre}
\begin{document}
	\maketitle
	\paragraph{Cos'è}Servizio di autenticazione che fornisce un server di autenticazione centralizzato la cui funzione è di autenticare gli utenti ai server e i server agli utenti. Anche i server sono autenticati tra di loro.
	
	\paragraph{Nota}Kerberos fa uso solo ed \ul{esclusivamente di crittografia simmetrica!!!}
	
	\paragraph{Requisiti Kerberos}Sicuro, trasparente, scalabile, affidabile.
	\paragraph{Attori}
	\begin{enumerate}
		\bitem{AS = authentication server} conosce gli hash delle password di tutti gli utenti e li memorizza in un database centralizzato e crea ogni volta una chiave simmetrica con il client sfruttando l'hash della password come seme da passare a funzioni pseudo random; condivide un'unica chiave con tutti i server;
		\bitem{TGS = ticket garanting server};
	\end{enumerate}
	\paragraph{Ticket}
	\begin{itemize}
		\item Ticket per entrare nel gioco che hanno una validità $Time_1$ ore $\to$ crittografia più forte;
		\item Ticket per utilizzare un servizio che hanno una validità $Time_2$ minuti $\to$ crittografia leggera;
	\end{itemize}
	$$Time_1>Time_2$$
	Man mano si usano ticket che durano sempre di meno.
	\paragraph{Authenticator}Chi sta operando e chi si dichiara sono la stessa persona.
\end{document}